\documentclass[notitlepage]{report}
\usepackage[left=0.5in, right=0.5in, top=0.5in, bottom=0.5in]{geometry}
\usepackage[parfill]{parskip}

\usepackage{graphicx}
\graphicspath{ {C:/Users/PaulB/Documents/msba/CIS_508/images/assignment_one/} }

\usepackage{titling}
\usepackage{lipsum}
\usepackage{float}
\usepackage{setspace}

\pretitle{\begin{center}\Huge\bfseries}
\posttitle{\par\end{center}\vskip 0.5em}
\preauthor{\begin{center}\Large\ttfamily}
\postauthor{\end{center}}
\predate{\par\large\centering}
\postdate{\par}
\doublespacing

\pagenumbering{gobble}

\title{CIS 508 Individual Assignment \#2}
\author{Paul Bernert}
\date{\today}
\begin{document}

\maketitle
\thispagestyle{empty}

\section*{To Do List}

\qquad \textbf{Task One:} Perform hyper-parameter tuning of the two classifiers by changing at least three different hyper-parameters of each classifier.

\qquad \textbf{Task Two:} Use both random and grid search for hyper-parameter tuning. Also use cross-validation.

\qquad \textbf{Task Three:} Discuss what hyper-parameter ranges are best for each classifier for these two problems.

\qquad \textbf{Task Four:} Compare and evaluate the different classifiers using the test-set and find the best classifier for criterion (e.g. accuracy, precision, recall, F1-score). 

\section*{Solution}
The objective of this homework is to move away from exclusively using the \textit{DecisionTreeClassifier} function within Scikit-learn and try other machine learning functions. This week, we will focus on hyper-parameter tuning using several different sklearn functions, including the: \textit{X} function, \textit{Y} function, and \textit{Z} function. 

\subsection*{Task One}

\subsection*{Task Two}

\subsection*{Task Three}

\subsection*{Task Four}

\end{document}